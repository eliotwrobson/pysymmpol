\documentclass[a4paper,10pt]{amsart}

\usepackage[backend=biber, style=alphabetic,
sorting=ynt]{biblatex}
\bibliography{../paper.bib}

\begin{document}

\title[PySymmPol]{PySymmPol: Symmetric Polynomials in Python}

\author{Thiago Araujo}

\address{\noindent 
Institute for Theoretical Physics, UNESP-Universidade Estadual Paulista
R. Dr. Bento T. Ferraz 271, Bl. II, Sao Paulo 01140-070, SP, Brazil\\
and
Institute of Physics\\
Universidade de S\~ao Paulo\\ 
Rua do Matão Travessa 1371, 05508-090\\
São Paulo, SP. Brazil}

\email{\texttt{tharaujo@if.usp.br}{tharaujo@if.usp.br}}

\keywords{python, symmetric polynomials}
\date{\today}

\maketitle

\setcounter{tocdepth}{1}
\tableofcontents


\section{Summary}

\textbf{PySymmPol}
is a Python package designed for efficient manipulation of
symmetric polynomials. It provides functionalities for working with
various types of symmetric polynomials, including elementary,
homogeneous, monomial symmetric, (skew-) Schur, and Hall-Littlewood
polynomials. In addition to polynomial operations, \textbf{PySymmPol} offers
tools to explore key properties of integer partitions and Young
diagrams, such as transposition, Frobenius coordinates, characters of
symmetric groups and others. 

This package originated from research conducted in the realm of
integrable systems applied to string theory and the AdS/CFT
correspondence. \textbf{PySymmPol} aims to facilitate computational tasks
related to symmetric polynomials and their applications in diverse
fields.

Link to the github repository: \texttt{github.com/thraraujo/pysymmpol}.

\section{Statement of need}

Symmetric polynomials play a crucial role across various domains of
mathematics and theoretical physics due to their rich structure and
broad applications. They arise naturally in combinatorics, algebraic
geometry, representation theory, and mathematical
physics~\cite{Macdonald:1998, Fulton:2004}. These polynomials encode
essential information about symmetries and patterns, making them
indispensable in the study of symmetric functions and their
connections to diverse mathematical structures. Moreover, symmetric
polynomials find extensive applications in theoretical physics,
particularly in quantum mechanics, statistical mechanics, and quantum
field theory~\cite{Babelon:2003, Korepin:1993, Marino:2005,
  Wheeler:2010}.  Their utility extends to areas such as algebraic
combinatorics, where they serve as powerful tools for solving
combinatorial problems and understanding intricate relationships
between different mathematical objects. Thus, tools and libraries like
PySymmPol provide researchers and practitioners with efficient means
to explore and manipulate symmetric polynomials, facilitating
advancements in both theoretical studies and practical applications.
 

\section{Main features and functionalities}

\textbf{PySymmPol} has several modules in two main packages. 
\begin{enumerate}
\item Partitions
\item Polynomials
\end{enumerate}
Let us briefly explain them. 

\subsection{Integer Partitions and Young Diagrams}

For integer partitions, we employ two primary representations. The
first is the conventional representation as a monotonic decreasing
sequence, denoted as $\lambda = (\lambda_1, \dots, \lambda_m)$, where
$\lambda_i \geq \lambda_{i+1}$. This representation enables basic
manipulations of integer partitions, including determining the number
of rows, columns, boxes, diagonal boxes, and Frobenius coordinates
within the associated Young Diagram.

Additionally, Young diagrams are represented as cycles within the
symmetric group $\mathfrak{S}_N$, denoted as $\lambda = (1^{k_1}
2^{k_2} \dots)$, where $N = \sum_i\lambda_i = \sum_j j k_k$. These
cycles serve as representatives of the conjugacy class of
$\mathfrak{S}_N$, and as such, we refer to them as conjugacy class
vectors. This alternative representation offers insights into the
structural properties of integer partitions and facilitates their
manipulation within the framework of group theory.

Our keen interest in exploring these representations stems
from their relevance in the realm of two-dimensional Conformal Field
Theories (CFTs). In this context, bosonic states are labeled by
conjugacy class vectors, highlighting the importance of
understanding and manipulating such representations. Conversely,
fermions are represented using the standard representation,
emphasizing the need to bridge the conceptual gap between these
distinct frameworks. Investigating these representations not only
sheds light on the fundamental properties of fermion states within
CFTs but also provides valuable insights for theoretical developments
in quantum field theory and related fields.

Some methods to deal with these bosonic and fermionic states are
also available in the package, and the \textrm{ACCELASC}
algorithm~\cite{Kelleher:2009} greatly improved the
speed of these methods.


\subsection{Symmetric Polynomials}

One of the main goals of the package is to provide the definitions of the 
polynomials in terms of the Miwa coordinates, or power sums, 
$$ t_j = \frac{1}{j} \sum_{i=1}^N x_i^j $$ where $\vec{x} = 
(x_1, \dots, x_N)$ and $\mathbf{t} = (t_1, t_2, \dots)$.

In terms of these coordiantes, the \emph{Complete Homogeneous} $h_n(\mathbf{t})$ 
and \emph{Elementary Symmetric Polynomials} $e_n(\mathbf{t}) = 
(-1)^n h_n(-\mathbf{t})$ are defined via 
$$ h_n(\mathbf{t}) = \sum_{k_1 + 2k_2+ \cdots = n} 
\frac{t_1^{k_1}}{k_1}\frac{t_2^{k_2}}{k_2} \cdots $$
From these expressions, we obtain the \emph{(skew-) Schur polynomials}
$s_{\lambda/\mu}(\mathbf{t})$, where 
$\lambda$ and $\mu$ are integer partitions, via \emph{Jacobi-Trudi identity}
$$ s_{\lambda/\mu} = \det_{p,q}(h_{\lambda_q - \mu_p - q + p}(\mathbf{t})) \; . $$

The \emph{Hall-Littlewood polynomials} are defined via
$$P_{\lambda}(x_1, \dots, x_N; Q) =
    \prod_{i\geq 0} \prod_{j=1}^{p(i)} \frac{1-Q}{1-Q^j}
    \sum_{\omega \in \mathfrak{S}_N} \omega\left( x_1^{\lambda_1}\cdots x_n^{\lambda_n}
    \prod_{i<j} \frac{x_i - Q x_j}{x_i - x_j} \right)$$
where $p(i)$ is the number of rows of size $i$ in $\lambda$, and $\mathfrak{S}_N$ is the 
symmetric group. The limit $Q=0$ in the Hall-Littlewood polynomials gives the Schur 
polynomials, while $Q=1$ returns the \emph{Monomial Symmetric Polynomials}
$m_\lambda(x_1, \dots, x_N)$.
        
For more information on these topics, see [REFERENCES].

\section{Acknowledgements}

I would like to thank Fapesp for finantial support, grant
\textbf{2022/06599-0}.  I would also like to thank the Free and Open
Source Software community.

\printbibliography
     
\end{document}
